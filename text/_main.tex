% AER-Article.tex for AEA last revised 22 June 2011
\documentclass[AER, draftmode]{AEA}

% The mathtime package uses a Times font instead of Computer Modern.
% Uncomment the line below if you wish to use the mathtime package:
%\usepackage[cmbold]{mathtime}
% Note that miktex, by default, configures the mathtime package to use commercial fonts
% which you may not have. If you would like to use mathtime but you are seeing error
% messages about missing fonts (mtex.pfb, mtsy.pfb, or rmtmi.pfb) then please see
% the technical support document at http://www.aeaweb.org/templates/technical_support.pdf
% for instructions on fixing this problem.

% Note: you may use either harvard or natbib (but not both) to provide a wider
% variety of citation commands than latex supports natively. See below.

% Uncomment the next line to use the natbib package with bibtex
% \usepackage{natbib}

% Uncomment the next line to use the harvard package with bibtex
% \usepackage[abbr]{harvard}

% This command determines the leading (vertical space between lines) in draft mode
% with 1.5 corresponding to "double" spacing.
\draftSpacing{1.5}

% Pandoc citation processing

\usepackage[colorlinks = true, allcolors = blue, linktoc = true, bookmarksdepth = 3]{hyperref}
\usepackage[format = hang, labelfont = bf, textfont = bf]{caption}
\usepackage{amsmath}
\usepackage{amssymb}
\usepackage{dsfont}
\usepackage{graphicx}
\usepackage{subcaption}
\usepackage{ragged2e}
\usepackage{floatrow}

\usepackage[style=chicago-authordate,backend = biber,related = false,sorting = nyt,includeall = false,giveninits = true,uniquename = mininit]{biblatex}
\addbibresource{references/biblio.bib}


% Placement of captions in tables and figures
\floatsetup[figure]{capposition=below}
\floatsetup[table]{capposition=top}

% Display all authors with surnames followed by first names
\DeclareNameAlias{author}{family-given}

% Do not print notes or language fields
\AtEveryBibitem{\clearlist{language}\clearfield{note}}

% Set autocite to parencite
\let\autocite\parencite

% Custom math symbols
\DeclareMathOperator{\arcsinh}{arcsinh}
\def\sym#1{\ifmmode^{#1}\else\(^{#1}\)\fi}

\begin{document}

\title{Title}
\shortTitle{Short title for headings}
\author{
    author1\\
    author2\thanks{
surname1: institution1, email1.
surname2: institution2, email2.
}
}

\date{\today}
\pubMonth{}
\pubYear{}
\pubVolume{}
\pubIssue{}
\issueName{}
\JEL{}
\Keywords{}

\begin{abstract}

\end{abstract}

\maketitle

\hypertarget{introduction}{%
\section{Introduction}\label{introduction}}

\hypertarget{literature}{%
\section{Literature}\label{literature}}

\hypertarget{institutional-background}{%
\section{Institutional background}\label{institutional-background}}

\hypertarget{data}{%
\section{Data}\label{data}}

\hypertarget{empirical-strategy}{%
\section{Empirical strategy}\label{empirical-strategy}}

\hypertarget{results-and-discussion}{%
\section{Results and discussion}\label{results-and-discussion}}

\hypertarget{robustness-checks}{%
\section{Robustness checks}\label{robustness-checks}}

\hypertarget{conclusion}{%
\section{Conclusion}\label{conclusion}}

\setlength{\baselineskip}{0pt}
\renewcommand*\MakeUppercase[1]{#1}\printbibliography[heading=bibintoc]

\appendix



\end{document}
